{\rtf1\ansi\ansicpg1252\cocoartf2757
\cocoatextscaling0\cocoaplatform0{\fonttbl\f0\fswiss\fcharset0 Helvetica;}
{\colortbl;\red255\green255\blue255;}
{\*\expandedcolortbl;;}
\paperw11900\paperh16840\margl1440\margr1440\vieww11520\viewh8400\viewkind0
\pard\tx720\tx1440\tx2160\tx2880\tx3600\tx4320\tx5040\tx5760\tx6480\tx7200\tx7920\tx8640\pardirnatural\partightenfactor0

\f0\fs24 \cf0 \\section\{Introduction\}\\label\{intro\}\
\\subsection\{Purpose\}\
In the context of education and development in the field of programming, students often face a series of challenges. The process of improving software development skills, both for beginners and more experienced students, requires a rigorous and structured approach.\\newline\
\
\\noindent The traditional method of learning based on theoretical lessons and assigned tasks can sometimes be limited in its effectiveness, as students may not have the opportunity to concretely apply what they have learned. Theory and practice must be integrated synergistically to ensure significant growth in software development skills.\
\\newline\
\
\\noindent CodeKataBattle (CKB) is an innovative response to these challenges in software learning and development. The CKB platform represents a revolutionary solution for students eager to enhance their programming skills. CKB is designed to transform the learning process into a collaborative and practical experience.\
\\newline\
\
\\noindent Thanks to CKB, students have the opportunity to engage in real code battles, solving programming exercises and overcoming a series of specific tests. These battles allow students to apply the theoretical knowledge they have acquired, putting into practice what they have learned through a series of specific challenges.\
\\newline\
\
\\noindent This document represents the RASD for the CodeKataBattle (CKB) system, providing a description focused on the system's requirements and specifications. It illustrates scenarios and use cases to detail the system's features, interactions with interested actors, and the limitations it is subject to.\
\
\\subsubsection\{Goal\}\
The system is characterized by the following goals:\
\\begin\{table\}[H]\
    \\begin\{tabularx\}\{\\textwidth\}\{cX\}\
        \\toprule\
        \\textbf\{G1\} & Educators create coding tournaments and battles              \\\\\
        \\textbf\{G2\} & Students register on the platform                           \\\\\
        \\textbf\{G3\} & Students form teams for coding battles  \\\\\
        \\textbf\{G4\} & Students participate in coding battles    \\\\\
        \\textbf\{G5\} & Students are notified about battles and tournaments \\\\\
        \\textbf\{G6\} & Projects can be partially evaluated by educators \\\\\
        \\textbf\{G7\} & Projects are partially evaluated in an automated way based on functional aspects, timeliness, and quality level of the sources  \\\\\
        \\textbf\{G8\} & Educators and students can see the rank of the battles and tournaments \\\\\
        \\\\ \\bottomrule\
    \\end\{tabularx\}\
\\end\{table\}}